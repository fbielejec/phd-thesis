\chapter{Generalised Linear Models with epoch structure\label{chap:five}}%PL: I would recommend to refer to mixed-effects modeling instead of GLM, see below

\begin{remark}{Outline}
In this chapter we discuss how this work can be connected to a Bayesian implementation of Generalised Linear Models presented by \citet{Lemey2014}, in order to couple the changes of substitution rate parameters over time to the changes in external co-variates.
\end{remark}

\section{Introduction}

% In those cases an interesting solution might be to couple the unknown transition-times between epochs to external covariates for which the change-points are known, such as  the fluctuations in population size recovered by the Bayesian skyline plot model \citep{Drummond2005}.
% %GB: why the Bayesian skyline and not the skyride or skygrid? What's so specific about the skyline that it lends itself to this more than the other demographic priors?
% This approach is being investigated at the time of writing this thesis. %, yet no publishable results are currently availiable 
% % DPPs

\section{Methods}


% An interesting direction for further research is to couple epoch-specific parameters to other external covariates to inform the inference.
In the Bayesian framework this could be achieved by formulating a hierarchical phylogenetic model \citep{Edo-Matas2011}, where one put "hyperpriors" on the parameters of prior distributions to avoid over-parametrisation of the model.
%PL: thanks for referring to our GLM work for this, but the Edo-Matas2011 work already lays out all the elements we need for this: it uses hierarchical phylogenetic modeling (and you can extend the discussion on this a bit as this has some history outside of phylogenetics, and Marc basically introduced them into phylogenetics). Hierarchical modeling provides the 'random' effects, but the Edo-Matas2011 work also introduces 'fixed' effects, for which we indeed use a model averaging approach using BSVSS and quantify their support (based on the indicators) and contribution (based on the coefficients or effect sizes). So, this is essentially mixed modeling. I don't think it was referred to this as such in the Edo-Matas2011 work, so it would be nice to discuss this as such here. Note, that mixed effects modeling is also what we have used for Bram's rate variability modeling in his HIV transmission chain, and this also a direction we would like to take with your codon model developments (modeling some random branch-specific variability while testing particular selection hypotheses through fixed effects). All good points to make here, or when this part turns into a chapter, or later in the discussion!
but it also introduces fixed effects
\citet{Lemey2014} propose a model which extends the Generalized Linear Models (GLM) of \citet{Nelder1972} to the Bayesian phylogenetic framework.
Every instantaneous rate $q_{ij}$, an entry of $K \times K$ generator matrix $\mathbf{Q}$, is parametrized as a log-linear function of the set of predictors $\mathbf{X}=\left( \mathbf{x_{1}},\ldots,\mathbf{x_{P}}\right)$, where each predictor $\mathbf{x_{p}}$ is characterized by its own rate matrix such that:

\begin{equation}
log(q_{ij})=\beta_{1}\delta_{1}x_{i,j,1}+\ldots+\beta_{P}\delta_{P}x_{i,j,P}.
\label{eq:glm}
\end{equation}

Coefficients $\beta=\left(\beta_{1},\ldots,\beta_{P}\right)$ quantify the contribution of a single predictor to the overall rate, and $\left(\delta_{1},\ldots,\delta_{P}\right)$ are binary indicators that decide whether a predictor is included or excluded from the model via a Bayesian stochastic search variable selection (BSSVS) procedure \citep{Kuo1998,Lemey2009}.

For the within-host HIV evolution example analyzed in Chapter~\ref{chap:epoch}, \citet{Shankarappa1999} report both viral load and CDT4 cell count data at different time points coinciding with most sequence samplings, which could then be used as a set of potential predictors of epoch-parameter variability, and with the epoch structure %formulized 
formulated 
for that model we could infer how the linear effect of each of the covariates changes over time.
If $\mathbf{x_{p}}$ is a single predictor we can use the epoch structure to divide its time-domain into contiguous intervals.
By fitting GLM models in those intervals a piecewise effect of predictors $\mathbf{x_{p}}$ can be obtained in each epoch. 
One could begin with standard piecewise-linear changes or approximate e.g. polynomial or other complex functions of rate-change using the approach presented in Subsection~\ref{sub:nonlinear}.
%PL: of course, you will deal with this when it this turns into a chapter, but it is important to note that we would like to apply this with codon models (GY94 and MG94) to investigate the changing dynamics of selection throughout HIV infection.


%PL: I would propose the following additional discussion section, something that Nidia started to work on 
%\subsection{Combining epoch modeling with graph hierarchies for Influenza phylogeography}
%Here you can discuss that we only briefly explored the epoch application to capture seasonal influenza migration patterns in our chapter. We can make the epoch structure as complex as we want, but it of course introduces a lot of parameters to estimate, and we have already difficulties informing standard phylogeographic models with a single location state observation at the tips of the tree. One way to alleviate the over-parameterization is by sharing information across epoch parameters using hierarchical modeling (discussed above). The other one is by doing BSSVS to shrink the number of parameters, which uses 0,1-indicators to augment the CTMC state space. Cybis et al have recently suggested a hierarchical modeling approach for graphs defined by these rate indicators. So, it would be useful to examine iowa combination of epochs and hierarchical graphs these perform for this problem and what patterns they can extract, and in general, what the best way would be to model the seasonal dynamics. For example we know through the 2014 GLM work that the rates follow air traffic, so perhaps we can fix those to passenger flux between locations and only estimate when we have to turn on or off those rates using the rate indicators with their hierarchical structure. Talk to me or Nidia if you need more info


\section{Results}


\section{Discussion}


