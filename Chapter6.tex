\chapter{General discussion and future directions}

In this concluding chapter several aspect sof the presented work are put into a wider perspective.
Much of the presented work resolves around relaxing the standard phylogenetic assumptions in search of more biologically plausible models. 
In that light we view our work on the time-heterogenous modelling as presented in Chapter \ref{chap:epoch} as an introductory step towards a broader range of models.
We begin with highlighting the possible extension in which time changes in a non-linear fashion.
We then discuss how this work can be connected to the work on Generalised Linear Models presented by \citet{Lemey2014} in order to couple the changes of substitution rate parameters over time to the changes in external co-variates.
Finally we talk about possibility to infer both the time and th enumber of change-points, or transition-times, in the epoch model specification vi aa promising family of priors driven by the Dirichlet process \citep{Ferguson1973}. 

% much of presented work is software
% simulation
Another substantial portion of the work presented in this thesis is devoted towards development of an easy-to-use software.
In that light we talk about the continued effort to support and extend the $\pi$BUSS simulation software \citep{bielejec2014} with an ever-growing array of phylogenetic and population models.

%visualising
% difussion
% spatio-temporal
Highly dimensional estimates that result from Bayesian inference of viral spread in both time and space require dedicated software that is capable of producing visulaisations that are both visually pleasing and insightfull.
We discuss the future directions that the next releases of the SPREAD software \citep{Bielejec2011} will take, ensuring that it keeps providing it's users with intuitive and user-friendly interfaces as well as access to a vast bulk of possible visualizations.


\section{Extending the Epoch model}

In Chapter \ref{chap:epoch} we presented our model of time-heterogeneous substitution model, for which rates of evolution remain constant across all lineages in any given epoch, but vary between epochs.
We have demonstrated the validity of the method using simulations, and it's applicability to the real HIV and Influenza data-sets.
We have stressed that by implementing the model as part of Bayesian framework we are able to incorporate uncertainty in the tree reconstruction into our analysis, and thus avoid the need to fix the tree topology.

\subsection{Generalised Linear Models with epoch structure}

An interesting direction for further research is to couple epoch-specific parameters to other external covariates to inform the inference.
In the Bayesian framework this could be achieved by formulating a hierarchical phylogenetic model \citep{Edo-Matas2011}, where one put "hyperpriors" on the parameters of prior distributions to avoid over-parametrisation of the model.
\citet{Lemey2014} propose a model which extends the Generalized Linear Models (GLM) of \citet{Nelder1972} to the Bayesian phylogenetic framework.
Every instantaneous rate $q_{ij}$, an entry of $K \times K$ generator matrix $\mathbf{Q}$, is parametrized as a log-linear function of the set of predictors $\mathbf{X}=\left( \mathbf{x_{1}},\ldots,\mathbf{x_{P}}\right)$, where each predictor $\mathbf{x_{p}}$ is characterized by it's own rate matrix such that:

\begin{equation}
log(q_{ij})=\beta_{1}\delta_{1}x_{i,j,1}+\ldots+\beta_{P}\delta_{P}x_{i,j,P}.
\label{eq:glm}
\end{equation}

Coefficients $\mathbf{\beta}=\left(\beta_{1},\ldots,\beta_{P}\right)$ quantify the contribution of single predictor to the overall rate, and $\left(\delta_{1},\ldots,\delta_{P}\right)$ are binary indicators that decide whether a predictor is removed or excluded from the model via Bayesian stochastic search variable selection (BSSVS) procedure \citep{Kuo1998,Lemey2009}.

For the within-host HIV evolution example analyzed in Chapter \ref{chap:epoch} \citet{Shankarappa99} report both viral load and CDT4 cell count data, which could be then used as a set of potential predictors, and with the epoch structure formulized for that model we could inferr how the linear effect of each of the covariates changes over time.

\subsection{Approximating non-linear functions of rate change in time}

For some classes of problems the substitution rate might vary not linearly, at the defined change-points, but rather as some arbitrarily complex function of time. 
Such functions may be difficult or computationally demanding to fit. 
Let us consider an interval $\left[0,T\right]$ and let elements of substitution rate matrix vary independenly as an integrable function of time, such that $\mathbf{Q}=\mathbf{Q}(t),\; t\in\left[0,T\right]$. 
Finite-time transition probabilities can be then calculated as: 

\begin{equation}
\ensuremath{\mathbf{P}}(r,T)=exp\left(r\int_{0}^{T}\mathbf{Q}(t)dt\right).\label{eq:rodrigo}
\end{equation}

\noindent
\citet{Rodrigo2008} propose a numerical approximation to $\int_{0}^{T}\mathbf{Q}(t)dt$. 
By taking points $t_{0,}t_{1},\ldots t_{n}\in[0,T]$ such that $0=t_{0}<t_{1}<\ldots<t_{n}=T$ and dividing the interval $\left[0,T\right]$ into sub-intevals $\left[t_{i-1},t_{i}\right],\; i=1,\ldots n$ of length $\triangle t_{i}=t_{i}-t_{i-1}$ we have that:   

\begin{equation}
\int_{0}^{T}\mathbf{Q}(t)dt\approx\underset{i=0}{\overset{n}{\sum}}\mathbf{Q}(t_{i}^{*})\cdot\triangle t_{i},\; t_{i}^{*}\in[t_{i-1},t_{i}].\label{eq:approx}
\end{equation}

\noindent
From \ref{eq:rodrigo} and \ref{eq:approx}, by the definition of the Riemannian integral we have that $\mathbf{P}(r,T)=\underset{n\rightarrow\infty}{lim}\underset{i=0}{\overset{n}{\prod}}\text{exp}\left(r\mathbf{Q}(t_{i}^{*})\cdot\triangle t_{i}\right)$, given that rate matrices $\mathbf{Q}(t_{i}^{*})$ commute 
and are closed with respect to the matrix multiplication.
\citet{Sumner2012} formalize the problem of multiplicative closure of the Markov models. 
Given those regularity conditions, we can perform a numerical approximation of any function of rate change by using the epoch time-discretization, and by partitioning the time interval into a fine grid of intervals. 

\subsection{Evaluating number and placement of change-points in time}

Both problems presented in Chapter \ref{chap:epoch} i.e. HIV within-host evolution before and after progression time and seasonal influenza migration represent hypothesis for which the number and placement of the transition times is known.
There might however be a class of problems for which it remains interesting to investigate possible extensions that estimate the number and position of change points.
First approach that comes to mind is to introduce priors on the number and location of the quantities and integrate over all possible values using the MCMC framework.
This straight-forward approach will however inflate the variance of the epoch-specific parameters and for some problems, 
like for example viral diffusion between discretely sampled geographical locations, the sparseness of availiable data might be a factor impeding any accurate inference.
In those cases an interesting solution would be to couple the unknown transition-times between epochs to external covariates for which the change-points are known, such as  the fluctuations in population size recovered by the Bayesian skyline plot model \citep{Drummond2005}.
% DPPs
Finally in Chapter \ref{chap:dpp} we talk about an interesting class of non-parametric prior distributions, the so called Dirichlet Process Priors (DPP). 
Although we mainly pursue the inference of lineage-specific parameters of codon models, it is interesting to note that the same class of priors could be used to inferr the number and placement of transition-times of the epoch model.

At the time of writing this chapter we are actively testing these possibilities, yet exacly how accurate the inference of the transition-times can be, and how we can quantify the predictive value of those covariates remains an open question for follow-up studies.


\section{pibuss, pibuss, pibuss}

With the exclusion of empirical models of evolution, most phylogenetic methods are built on mathematical theories that predict outcomes conditional on some specified assumptions.
How well these models fit the real data, and how robust they are to the violations of these assumptions remains an open question.
Furthermore the complex biologicla systems being the topic of phylogentic studies undergo many processes, which interact and are difficult to accurately model or distinguish from the background noise in the data.
Rapidly evolving viruses, which are the main interest of the research presented in this thesis, undergo mutaion, natural selection and spatial diffusion.
Therefore for most of epidemiological the molecular data-sets the true underlying evolutionary process that generated them is unknown.
These factors stress the need for development of flexible simulation software that can match the diversity, complexity and sheer amount of the real data that is currently being sampled.

% adding more models
Phylogenetic simulation software $\pi$BUSS presented in Chapter \ref{chap:pibuss} allows for fabricating evolution under a vast array of coalescent, amino-acid, nucleotide and codon substitution models, as well as diffusion models, combined with various molecular clock models.
In addition it offers an ability to apply different models across arbitrary partition schemes.
In future releases of $\pi$BUSS the development will be aimed at extending the array of availiable models even further.
The development effort will be aimed at matching the model richness available for inference in BEAST and supplying every model with it's simulation counterpart in $\pi$BUSS.

% gui for more complex models
In particular the models specification will be availiable via the user-friendly Graphical User Interface (GUI), which allows to setup a simulation by choosing models from drop-down menus, partitioning data in tables and parsing parameters from text fields in an intuitive fashion.
Time-heterogenous models, for which evolutionary parameters change both across different lineageas (see Chapter \ref{chap:dpp}) as well as models of pan-lineage change (see Chapter \ref{chap:epoch}), which form a vital part of this thesis will have a dedicated panels, where users can formulate them without the need to manually edit XML files.

% beauti integration
In that sense $\pi$BUSS in addition to it's Command Line Interface for scripting purposes and XML parsers availiable via BEAST core implementation is similar to BEAUti software, which helps in setting analysis via user-friendly GUI.
We will follow along that path, by providing even closer integration between the two programs and facilitating generation of joint simulation-analysis XML documents in which fabricated data is instantly passed to BEASTs XML parsers for inference.

\section{Future prospects for visualising viral diffusion}
















