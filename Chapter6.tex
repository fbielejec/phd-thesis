\chapter{General discussion and future directions}

In this concluding chapter several aspect sof the presented work are put into a wider perspective.
Much of the presented work resolves around relaxing the standard phylogenetic assumptions in search of more biologically plausible models. 
In that light we view our work on the time-heterogenous modelling as presented in Chapter \ref{chap:epoch} as an introductory step towards a broader range of models.
We begin with highlighting the possible extension in which time changes in a non-linear fashion.
We then discuss how this work can be connected to the work on Generalised Linear Models presented by \citet{Lemey2014} in order to couple the changes of substitution rate parameters over time to the changes in external co-variates.
Finally we talk about possibility to infer both the time and th enumber of change-points, or transition-times, in the epoch model specification vi aa promising family of priors driven by the Dirichlet process \citep{Ferguson1973}. 

% much of presented work is software
% simulation
Another substantial portion of the work presented in this thesis is devoted towards development of an easy-to-use software.
In that light we talk about the continued effort to support and extend the $\pi$BUSS simulation software \citep{bielejec2014} with an ever-growing array of phylogenetic and population models.

%visualising
% difussion
% spatio-temporal
Highly dimensional estimates that result from Bayesian inference of viral spread in both time and space require dedicated software that is capable of producing visulaisations that are both visually pleasing and insightfull.
We discuss the future directions that the next releases of the SPREAD software \citep{Bielejec2011} will take, ensuring that it keeps providing it's users with intuitive and user-friendly interfaces as well as access to a vast bulk of possible visualizations.


\section{Extending the Epoch model}
% extending the epoch model (estimable placements of transition times - connects with DPP), Allen Rodrigos comments, GLM's with splines, 























\section{pibuss, pibuss, pibuss}



\section{Future prospects for visulising viral diffusion}
















