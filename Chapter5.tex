\chapter{Branch specific codon models}

\section{Introduction}

First computationally trackable models of codon substitution were independently proposed by \cite{Muse1994} and \cite{Goldman1994} and published in the same issue of Molecular Biology and Evolution (MBE).
For codon models the non-stop codon triplet $n_{1}n_{2}n_{3}$ is considered as a smallest unit of evoluton.
There are $4^3$ possible triplets minus three stop-codons, resulting in a state space size of $61$ codons.
The standard assuption of independence is used, i.e. the substitutions at these three codon positions occur independently, thus only a single change per triplet can occur at a given time.
There are several advantages in using codon models over nucleotide based substitution models.
Not all DNA positions evolve at the same rate, with non-synonymous substitutions occuring more frequently the synonymous substitutions.
Although this problem can be mitigated to some extent by using codon-positioned nucleotide substitution models, the fast evolving positions and the state space limited to 4 character alphabet leads to biased estimates of long evolutionary distances, as portrayed in the previous chapter (\ref{sub:deepRoot}).

% differences between models
% GY style models
The model proposed by \cite{Goldman1994} is characterized by a substitution rate matrix with following entries:

\begin{equation}
q_{ij}^{GY94}=\begin{cases}
\pi_{j} & \text{if \ensuremath{i\rightarrow j} is a synonymous transversion}\\
\kappa\cdot\pi_{j} & \text{if \ensuremath{i\rightarrow j} is a synonymous transition}\\
\omega\cdot\pi_{j} & \text{if \ensuremath{i\rightarrow j} is a non-synonymous transversion}\\
\omega\cdot\kappa\cdot\pi_{j} & \text{if \ensuremath{i\rightarrow j} is a non-synonymous transversion}\\
0 & \text{otherwise}
\end{cases},
\label{eq:gy94}
\end{equation}

\noindent
where parameter $\kappa$ denotes the transition/transversion rate ratio, parameter $\omega$ denotes the non-synonymous/synonymous
rate ratio and $\pi_j$ denotes the equilibrium frequency of codon triplet $j$.
Parameter $\kappa$ and $\pi_j$ can be though of as controlling the CTMC process at the DNA level, while $\omega$ parameter characterizes the selection on non-synonymous substitutions.

Different flavours of the GY94 model differ in the composition of the equilibrium codon frequency parameter $\pi_{j}$.
One approach is to model the codon frequencies as each having the same long-time frequency of apperaring and such a model is reffered to as the F-equal.
In the GY94-F$1\times4$ model, each codon position has 


codon frequencies are expected from the four nucleotide frequencies, giving three free parameters, as they need to some to one.

a separate set of nucleotide propensity parameters to each of the three codon positions.



For the GY94-F$3\times4$ model frequencies are parametrized according to three sets of nucleotide frequencies for the three codon positions, resulting in nine free parameters.
Finally in the GY94-F61 model every codon triplet has it's own frequncy parameter, with all parameters summing to one, resulting in 60 free parameter that need to be estimated.





















\section{Methods}

% Dirichlet distribution

% Dirichlet process
% - different constructions


We use Dirichlet process prior (DPP) to model branch specific codon parameters.
We will use $\mathbf{z}$ to denote a vector of branch category assignments.
Each branch receives a category $z_{i},\; i\in\left\{ 1,\ldots K\right\}$, where $K\in\left\{ 1,\ldots,2N-1\right\}$ denotes the a number of categories.
We can thus view $\mathbf{z}=\left(z(1),\ldots,z(2N-1)\right)$ as a vector of mappings between branches and categories.
 
As an example let us consider a fixed topology $\mathbf{F}$ with $N=2$ taxa and $K=2$ category classes.
Then one possible realization of the vector of assignments would be:

$$\mathbf{z}=\left(z(1)=1,\; z(2)=1,\; z(3)=2\right),$$
 
\noindent
meaning that branches one and two belong to the same category 1, and that the third branch belongs to the category 2.
%TODO: plot

Under our model both the number of categories $K$ and category assignments $z_{i}$ are random variables, controlled by a DPP with a concentration parameter $\gamma$ and a base distribution $B$.
They are drawn from the probability distribution:

\begin{equation}
P(\mathbf{z},K|\gamma,2N-1)=\gamma^{K}\cdot\frac{\underset{k=1}{\overset{K}{\prod}}\left(\eta_{k}-1\right)!}{\underset{i=1}{\overset{2N-1}{\prod}}\left(\gamma+i-1\right)},
\label{eq:dpp1} 
\end{equation} 
 
\noindent 
where $\eta_{k}$ denotes the number of sites assigned to category $k$.

By integrating (\ref{eq:dpp1}) over all possible branch assignments for $k$ categories we get the prior probability of having a given number of categories:

\begin{equation}
P(K|\gamma,2N-1)=\gamma^{K}\cdot\frac{S_{1}(2N-1,K)}{\underset{i=1}{\overset{2N-1}{\prod}}\left(\gamma+i-1\right)},
\label{eq:dpp2} 
\end{equation} 

\noindent
where $S_{1}(2N-1,K)$ is the Stirling number of the first kind. 

% computing the number of ways to 
% \begin{equation}
% TODO: Stirling number of the first kind
% \label{eq:stirling} 
% \end{equation} 

Special care needs to be taken when choosing the value of the concentration parameter $\gamma$ of the DPP. 
Parameter $\gamma$ controls the 
Small values will lead to a fewer, yet more populated categories, large values will result in more categories being occupied by less branches.
%TODO: hierarchical prior on \gamma?

Once all the branches are mapped to a category, the branch specific parameters $\theta_{k}\; k\in\left\{ 1,\ldots,K\right\}$ are drawn $K$ times from their respective parametric probability distribution $G$.


\section{Results}




\section{Discussion}



