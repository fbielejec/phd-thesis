\chapter{Branch specific codon models}

\section{Introduction}

First computationally trackable models of codon substitution were independently proposed by \cite{Muse1994} and \cite{Goldman1994} and published in the same issue of Molecular Biology and Evolution (MBE).



































\section{Methods}
% TODO implement (14) from Huelsenbech 2007 ((6) Huelsenbeck 2005, (page 3) Heath 2011)
We use Dirichlet process prior (DPP) to model branch specific codon parameters.
We will use $\mathbf{z}$ to denote a vector of branch category assignments.
Each branch receives a category $z_{i},\; i\in\left\{ 1,\ldots K\right\}$, where $K\in\left\{ 1,\ldots,2N-1\right\}$ denotes the a number of categories.
We can thus view $\mathbf{z}=\left(z(1),\ldots,z(2N-1)\right)$ as a vector of mappings between branches and categories.
 
As an example let us consider a fixed topology $\mathbf{F}$ with $N=2$ taxa and $K=2$ category classes.
Then one possible realization of the vector of assignments would be:

$$\mathbf{z}=\left(z(1)=1,\; z(2)=1,\; z(3)=2\right),$$
 
\noindent
meaning that branches one and two belong to the same category 1, and that the third branch belongs to the category 2.
%TODO: plot

Under our model both the number of categories $K$ and category assignments $z_{i}$ are random variables, controlled by a DPP with a concentration parameter $\gamma$ and a base distribution $B$.
They are drawn from the probability distribution:

\begin{equation}
P(\mathbf{z},K|\gamma,2N-1)=\gamma^{K}\cdot\frac{\underset{k=1}{\overset{K}{\prod}}\left(\eta_{k}-1\right)!}{\underset{i=1}{\overset{2N-1}{\prod}}\left(\gamma+i-1\right)},
\label{eq:dpp1} 
\end{equation} 
 
\noindent 
where $\eta_{k}$ denotes the number of sites assigned to category $k$.

By integrating (\ref{eq:dpp1}) over all possible branch assignments for $k$ categories we get the prior probability of having a given number of categories:

\begin{equation}
P(K|\gamma,2N-1)=\gamma^{K}\cdot\frac{S_{1}(2N-1,K)}{\underset{i=1}{\overset{2N-1}{\prod}}\left(\gamma+i-1\right)},
\label{eq:dpp2} 
\end{equation} 

\noindent
where $S_{1}(2N-1,K)$ is the Stirling number of the first kind. 

% computing the number of ways to 
% \begin{equation}
% TODO: Stirling number of the first class
% \label{eq:stirling} 
% \end{equation} 

Once all the branches are mapped to a category, the branch specific parameters $\theta_{k}\; k\in\left\{ 1,\ldots,K\right\}$ are drawn $K$ times from their respective parametric probability distribution $G$.


\section{Results}




\section{Discussion}



